\chapter{Java通信接口}

\section{消息格式}
xserver与java之间的RPC通信建立在tcp之上,并通过protobuf编码的。结构如图:

\begin{small}
\begin{figure}[h]
\centering
\scalebox{0.95}{
    \begin{tikzpicture}[x=1pt,y=1pt,auto,start chain=going below]
        \tikzstyle{every node}=[fill=none,rectangle,minimum height=18pt,draw,node distance=0pt,inner sep=0pt]
        \tikzstyle{note}=[draw=none,above,opaque,fill=none]
        \tikzstyle{hide}=[note,minimum width=0pt]
        \tikzstyle{state}=[inner sep=0pt,minimum height=18pt,minimum width=60pt]
        {
            \node[hide] at (-80, +20) {xserver};
            \node[hide] at (130, +20) {java};

            \draw[color=gray,thick](30, 44) rectangle (230,-140);
            \node[hide] at (+180, -133) {\textit{\footnotesize{XServerData2}}};

            \node[state] (x1) at (-80, 0) {protobuf};
            \node[state] (j1) at (+80, 0) {protobuf};

            \node[state,node distance=50] (x2) [below of=x1] {bytes};
            \node[state,node distance=50] (x3) [below of=x2] {message};

            \node[state,node distance=50] (j2) [below of=j1] {bytes};
            \node[state,node distance=50] (j3) [below of=j2] {message};

            \node[state,node distance=100] (j4) [right of=j1] {request};
            \node[state,node distance=25] (j5) [below of=j4] {response};
            \node[state,node distance=25] (j6) [below of=j5] {broadcast};
            \node[state,node distance=25] (j7) [below of=j6] {close};

            \path[<->,>=stealth',shorten >=1pt,auto,thick,very node/.style={font=\sffamily\small}]
                (x3) edge node[hide,above] {tcp} (j3)
                (x2) edge (x3)
                (x1) edge (x2)
                (j2) edge (j3)
                (j1) edge (j2)
                (j1) edge[->,dotted] (j4)
                (j5.west) edge[->,dotted] (j1)
                (j6.west) edge[->,dotted] (j1)
                (j7.west) edge[->,dotted] (j1);
        }
    \end{tikzpicture}
}
\caption{RPC过程数据包处理示意图}
\end{figure}
\end{small}

\subsubsection{tcp层数据包格式}

在tcp连接上传输的数据包由三部分组成,以发送序列化后长度为13565=0x34fd个字节的protobuf数据包为例:

\begin{itemize}
    \item [a.] 先是4字节魔数,值为 0xdeadbeaf;
    \item [b.] 再是4字节数据包长度,本例为 0x34fd;
    \item [c.] 最后是将全部数据包内容;
\end{itemize}

\begin{small}
\begin{figure}[h]
\centering
\scalebox{0.95}{
    \begin{tikzpicture}[x=1pt,y=1pt,transform shape,font=\sffamily\small]
        \tikzstyle{data} = [draw=black,minimum width=30, minimum height=30, node distance=30,fill=white]
        \tikzstyle{data0} = [data,fill=gray!25]
        \tikzstyle{data1} = [data,fill=gray!55]
        {
            \node[data0] (m1) {0xde};
            \node[data0] (m2) [right of=m1] {0xad};
            \node[data0] (m3) [right of=m2] {0xbe};
            \node[data0] (m4) [right of=m3] {0xaf};

            \node[data1] (s1) [right of=m4] {0x00};
            \node[data1] (s2) [right of=s1] {0x00};
            \node[data1] (s3) [right of=s2] {0x34};
            \node[data1] (s4) [right of=s3] {0xfd};

            \node[data] (b1) [right of=s4] {};
            \node[data] (b2) [right of=b1] {};
            \node[data,draw=none,fill=none] (b3) [right of=b2] {...};
            \node[data] (b4) [right of=b3] {};
            \node[data] (b4) [right of=b4] {};

            \draw[decorate,decoration={brace,amplitude=10}]
			    (370,-18) -- (230,-18) node[black,midway,yshift=-20] {\small{数据包内容(共0x34fd字节)}};
            \draw[decorate,decoration={brace,amplitude=10}]
			    (100,-18) -- (-10,-18) node[black,midway,yshift=-20pt] {\small{魔数=0xdeadbeaf}};
            \draw[decorate,decoration={brace,amplitude=10}]
			    (220,-18) -- (110,-18) node[black,midway,yshift=-20pt] {\small{数据包大小=0x34fd}};
        }
    \end{tikzpicture}
}
\caption{RPC过程数据包格式}\label{chap1:RPCdata}
\end{figure}
\end{small}

\section{消息类型}

\subsection{request消息}

java 程序从xserver 收到的消息格式定义为 request,所有字段含义如下:

\begin{javacode}
public class XServerData {
    public static class Request {
        public int port;           # `{\songti xserver listen端口号}`
        public String code;        # `{\songti request过程操作类型:只有\{join,exit,call\}三种取值}`
        public int xid;            # `{\songti 对应的session id}`
        public String addr;        # `{\songti 客户端外网地址}`
        public boolean reliable;   # `{\songti 消息是否为reliable}`
        public double callback;    # `{\songti 客户端callback}`
        public Amf0Input input;    # `{\songti request调用内容}`
    }
\end{javacode}

使用时只需要阻塞的调用如下静态函数即可,具体实现请参见源代码:

\begin{javacode}
    public static Request parseRequest(InputStream is) throws Exception;
\end{javacode}

\subsubsection{join消息}
当客户端与xserver建立连接,无论有无进一步数据通信,xserver都会向后端RPC服务发送``join''消息。
则此时request只有如下四个字段有意义:

\begin{javacode}
    public static class Request {
        public int port;
        public String code = "join";
        public int xid;
        public String addr;
    }
\end{javacode}

\subsubsection{exit消息}
当客户端与xserver断开连接,包括客户端主动断开或者被xserver被动踢掉,xserver都会向后端RPC服务发送``exit''消息。

\begin{javacode}
    public static class Request {
        public int port;
        public String code = "exit";
        public int xid;
        public String addr;
    }
\end{javacode}

\subsubsection{call消息}
该过程是真正的客户端到后端服务的RPC过程。过程中全部字段都有意义:
\begin{javacode}
    public static class Request {
        public int port;
        public String code = "call";
        public int xid;
        public String addr;
        public boolean reliable;   # `{\songti false表示数据由proxySend2发出;否则为true}`
        public double callback;    # `{\songti != 0表示客户端有有效callback handler;否则为0}`
        public Amf0Input input;    # `{\songti != null表示有有效content数据;否则为null}`
    }
\end{javacode}

\subsection{response消息}

当服务端需要调用客户端的callback handler时,例如:服务端收到callback!=0的reqeust,并完成服务,现在将要返回调用结果给客户端时,
需要通过response接口来完成:

\begin{javacode}
    public static byte[] newResponse(int xid, byte[] data, double callback, boolean reliable) throws Exception;
\end{javacode}

该函数生成完整的response消息:魔数+消息大小+消息内容,参见图\ref{chap1:RPCdata}。
例如:
\begin{javacode}
    OutputStream os = socket.getOutputStream();
    os.write(XServerData2.newResponse(xid, data, callback, reliable));
    os.flush();
\end{javacode}

此外额外的参数说明如下:

\begin{itemize}
    \item [a.] data 表示发送的数据,例如序列化后的amf数据;可以为空,即null
    \item [b.] callback 表示客户端callback handler,通常来自 request
    \item [c.] reliable 表示消息是否由xserver保证消息可靠,如果为false,消息可能会由于网络问题造成丢失
\end{itemize}

{\bf{注意:reliable和unreliable消息不要频繁交替发送,
否则客户端可能会由于之前reliable消息的丢包重传造成unreliable消息的deliver延迟。}}

\subsection{broadcast消息}

当服务器需要向单个或者多个客户端广播消息时,需要使用broadcast消息接口:

\begin{javacode}
    public static byte[] newBroadcastMesssage(List<Integer> xids, byte[] data, boolean reliable) throws Exception;
\end{javacode}

使用方法以及注意事项与response过程相同。
不同的是,该过程消息会被deliver到NetConnection的回调函数上,如果是reliable消息,
回调函数为recvPull,否则为recvPull2。

\subsection{close消息}

当后端java主动断开客户端的连接时可以使用该接口:
\begin{javacode}
    public static byte[] newCloseMessage(List<Integer> xids) throws Exception;
\end{javacode}

xserver收到该函数请求后,会主动断开与客户端的连接。并想后端java发送code为``exit''的request消息。

